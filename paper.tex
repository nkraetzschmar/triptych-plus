\documentclass{article}

\usepackage[utf8]{inputenc}
\usepackage[T1]{fontenc}
\usepackage[english]{babel}
\usepackage[style=alphabetic,citestyle=alphabetic]{biblatex}
\usepackage{hyperref}
\usepackage[a4paper]{geometry}
\usepackage{amsmath}
\usepackage{amsfonts}
\usepackage{amsthm}
\usepackage[nolist]{acronym}
\usepackage{xifthen}
\usepackage{float}
\usepackage{graphicx}
\usepackage{xcolor}
\usepackage[capitalise,nameinlink]{cleveref}
\usepackage{chngcntr}

\numberwithin{equation}{section}
\counterwithin{figure}{section}
\newtheorem{theorem}{Theorem}[section]
\newtheorem{lemma}{Lemma}[section]
\newtheorem{corollary}{Corollary}[section]
\theoremstyle{definition}
\newtheorem{definition}{Definition}[section]

\newcommand\NB{\textbf{$\text{N}\!\!\text{B} \;$}}
\newcommand\N{\mathbb N}
\newcommand\Z{\mathbb Z}
\newcommand\G{\mathbb G}
\renewcommand\P{\mathcal P}
\newcommand\V{\mathcal V}
\newcommand\A{\mathcal A}
\renewcommand\H{\mathcal H}
\newcommand\from{\leftarrow}
\newcommand\Com{\text{Com}}
\newcommand\Enc{\text{Enc}}

\newcommand\set[2]{\Big\{ \, #1 \; \Big| \; #2 \, \Big\}}
\newcommand\prob[2]{\text{Pr} \Big[ \, #1 \; : \; #2 \, \Big]}

\newcounter{relationcntr}
\newcommand{\R}[1]{\refstepcounter{relationcntr}\mathcal{R}_{\therelationcntr}\label{#1}}
\newcommand{\Rref}[1]{\ensuremath{\mathcal{R}_{\ref{#1}}}}

\newenvironment{indentedblock}[1][0.5em]
{
	\par
	\leftskip#1\relax
}
{
	\par
}

\newenvironment{sigmaprotocol}[1][\textwidth]{%
	\newcommand\prover[3][$\P$]{
		\underline{\textbf{Prover} ##1:}
		\begin{indentedblock}
			##2
			\ifthenelse{\isempty{##3}}{}{\par Send ##3 $\to$ Verifier}
		\end{indentedblock}
		\vspace{1.5\parskip}
	}%
	\newcommand\verifier[3][$\V$]{
		\underline{\textbf{Verifier} ##1:}
		\begin{indentedblock}
			##2
			\ifthenelse{\isempty{##3}}{}{\par Send ##3 $\to$ Prover}
		\end{indentedblock}
		\vspace{1.5\parskip}
	}%
	\begin{minipage}{#1}
		\setlength{\parskip}{0.5\baselineskip}
}{
	\end{minipage}%
}

\setlength{\parindent}{0cm}
\setlength{\parskip}{0.5\baselineskip}

\addbibresource{bibliography.bib}


\title{\textbf{Triptych+}\\Even shorter linkable or accountable ring signatures using ring dependent commitment schemes}
\author{Nikolas Krätzschmar}
\date{\today}

\begin{document}

\newacro{EC}{Elliptic Curve}
\newacroplural{EC}[EC]{Elliptic Curves}
\newacro{SHVZK}{special honest-verifier zero-knowledge}

\maketitle

In this paper we present a novel linkable ring signature $\Sigma$--protocol, derived from \citetitle{triptych} \cite{triptych} by \citeauthor{triptych}, that achieves even shorter signatures without incurring any noticeable computational overhead.
We show that its security only relies on the same cryptographic assumptions as Triptych, namely hardness of the discrete logarithm problem and Diffie-Hellman problems.

Additionally, we provide an adaptation of our $\Sigma$--protocol into an accountable ring signature which also outperforms \citetitle{short_ring_signatures} \cite{short_ring_signatures} by \citeauthor{short_ring_signatures} in terms of size.

\section{Notation}
\label{notation}

\begin{definition}
	For a set $X$ we use the notation $x \from X$ to write an element $x \in X$ chosen uniformly at random.
	For an algorithm $\A$ we use the notation $x \from \A$ to write that element $x$ is output by $\A$.
\end{definition}

\begin{definition}
	Let $\G$ be a cyclic group of prime order $q$ with generator $G$ in which the discrete logarithm problem is hard and the decisional Diffie-Hellman assumption and inverse decisional Diffie-Hellman assumption \cite{dh_vars} hold.
	We use additive notation for the group operation on $\G$, as is the convention for \acp{EC}.

	\NB The inverse decisional Diffie-Hellman assumption is required for the way key images are constructed both in Triptych and in our novel construction.
\end{definition}

% \begin{definition}
% 	Throughout this paper, we will consistently use the following convention for indices to avoid confusion:
% 	\begin{itemize}
% 		\item All vector and matrix indices start at $0$
% 		\item $N=n^m$
% 		\item $i$ used for indices $\in \{ 0, ..., m-1 \}$
% 		\item $j$ used for indices $\in \{ 0, ..., n-1 \}$
% 		\item $k$ used for indices $\in \{ 0, ..., N-1 \}$
% 		\item $u$ used for indices $\in \{ 0, ..., d-1 \}$
% 	\end{itemize}
% \end{definition}

\begin{definition}[Kronecker delta function]\label{def:delta}
	Let the function $\delta : X \times X \to \{ 0, 1 \}$ be defined as the Kronecker delta function:
	$$
		\delta(x, y) = \begin{cases}
			1 \; & \text{if } x = y \\
			0 \; & \text{else}
		\end{cases}
	$$
\end{definition}


\clearpage


\section{Commitment to a selection matrix}
\label{commitment_mat}

First we re-iterate the $\Sigma$--protocol of Fig. 4 in \cite{short_ring_signatures}.
It proves that a commitment $B$ is a matrix Pedersen commitment to a special type of binary matrix, which will herein be refereed to as a selection matrix.

This $\Sigma$--protocol is both used as a sub-protocol in the accountable ring signature protocol of \cite{short_ring_signatures} and the basis of the linkable ring signature protocol of \cite{triptych}.
We will also use it as a sub-protocol in our ring signature construction.

\begin{definition}[Selection Matrix]
A selection matrix is herein defined as a binary matrix that has exactly one $1$ in each row and $0$s everywhere else.
So formally a matrix $a \in \Z_q^{m \times n}$, with $n < q$, is a selection matrix if and only if
$$\begin{matrix}
	\forall i \in \{ 0, ..., m-1 \}, j \in \{ 0, ..., n-1 \} : a_{i,j} \in \{ 0, 1 \} \\
	\land \; \forall i \in \{ 0, ..., m-1 \} : \sum_{j=0}^{n-1} a_{i,j} = 1
\end{matrix}$$
\end{definition}

\cref{fig:commitment_mat} is a $\Sigma$--protocol for the following relation \Rref{rel:commitment_mat}, so in essence that a commitment $B$ is a commitment to a selection matrix.
$$
\R{rel:commitment_mat} = \left\{
	(B, ((b_{i,j})_{i=0,j=0}^{m-1,n-1}, r_B)) \; \middle| \;
	\begin{matrix}
		(\forall i \in \{ 0, ..., m-1 \}, j \in \{ 0, ..., n-1 \} : b_{i,j} \in \{ 0, 1 \}) \\
		\land \; (\forall i \in \{ 0, ..., m-1 \} : \sum_{j=0}^{n-1} b_{i,j} = 1) \\
		\land \; B = \Com((b_{i,j})_{i=0,j=0}^{m-1,n-1}, r_B)
	\end{matrix}
\right\}
$$

\begin{theorem}\label{theorem:commitment_mat}
	The $\Sigma$--protocol in \cref{fig:commitment_mat} is perfectly complete, \ac{SHVZK}, and $3$-special sound.
\end{theorem}

The proof of \cref{theorem:commitment_mat} is given in appendix B.1 of \cite{short_ring_signatures}.


\begin{figure}[p]
	\centering
	\fbox{\begin{sigmaprotocol}[0.8\textwidth]
		\prover[$\P_{\Rref{rel:commitment_mat}}(B, ((b_{i,j})_{i=0,j=0}^{m-1,n-1}, r_B))$]{
			Select random $r_A, r_C, r_D, a_{i,j} \from \Z_q \quad \forall i \in \{ 0, ..., m-1 \}, j \in \{ 1, .., n-1 \}$ \\
			For all $i \in \{ 0, ..., m-1 \}$ let
			$$a_{i,0} = - \sum_{j=1}^{n-1} a_{i,j}$$
			Define
			$$\begin{aligned}
				A &= \Com((a_{i,j})_{i=0,j=0}^{m-1,n-1}, r_A) \\
				C &= \Com((a_{i,j}(1 - 2b_{i,j}))_{i=0,j=0}^{m-1,n-1}, r_C) \\
				D &= \Com((-a_{i,j}^2)_{i=0,j=0}^{m-1,n-1}, r_D)
			\end{aligned}$$
		}{$(A, C, D)$}

		\verifier[$\V_{\Rref{rel:commitment_mat}}(B, (A, C, D))$]{
			Select random $x \from \Z_q$
		}{$x$}

		\prover[$\P_{\Rref{rel:commitment_mat}}(x)$]{
			For all $i \in \{ 0, ..., m-1 \}, j \in \{ 1, .., n-1 \}$ define
			$$\begin{aligned}
				f_{i,j} &= x b_{i,j} + a_{i,j} \\
				z_A     &= x r_B + r_A \\
				z_C     &= x r_C + r_D
			\end{aligned}$$
		}{$(z_A, z_C, (f_{i,j})_{i=0,j=1}^{m-1,n-1})$}

		\verifier[$\V_{\Rref{rel:commitment_mat}}(z_A, z_C, (f_{i,j})_{i=0,j=1}^{m-1,n-1})$]{
			For all $i \in \{ 0, ..., m-1 \}$ let
			$$f_{i,0} = x - \sum_{j=1}^{n-1} f_{i,j}$$
			\par
			Accept if and only if:
			\begin{align}
				x B + A &= \Com((f_{i,j})_{i=0,j=0}^{m-1,n-1}, z_A) \\
				x C + D &= \Com((f_{i,j} (x - f_{i,j}))_{i=0,j=0}^{m-1,n-1}, z_C)
			\end{align}
		}{}
	\end{sigmaprotocol}}
	\caption{$\Sigma$--protocol for \Rref{rel:commitment_mat} (commitment to selection matrix) \cite{short_ring_signatures}}
\label{fig:commitment_mat}\end{figure}


\clearpage


\section{Ring Signature}
\label{ring}

First we provide a modification of Triptych \cite{triptych}, reducing it to a simple ring signature without linkability properties and using a dynamic base $I$ rather than the constant generator $G$ for public keys.
Then in section \ref{linkable}, using this initial construction as a sub-protocol, we re-introduce key images in a more efficient way, somewhat similar to how accountability is achieved in \cite{short_ring_signatures}.
This construction can also be used to construct accountable ring signatures (section \ref{accountable}) that are even shorter than those of \cite{short_ring_signatures}.

\cref{fig:ring} provides a $\Sigma$--protocol for the base $I$ ring signature relation \Rref{rel:ring}:
$$\R{rel:ring} = \set{((I, (R_k)_{k=0}^{N-1}), (\ell, r))}{R_\ell = rI}$$

It uses the following base $n$ decomposition of integers $k \in \{ 0, ..., N-1 \}$, where $N = n^m$.

\begin{definition}[$n$-ary decomposition]
	Let $N = n^m$ for some $n, m \in \N$.
	For any $k \in \{ 0, ..., N-1 \}$ define the $n$-ary decomposition of $k$ as its unique representation as base $n$ digits $(k_0, ..., k_{m-1})$ such that
	$$\forall i \in \{ 0, ..., m-1 \} : \; k_i \in \{ 0, ..., n-1 \} \quad \text{and} \quad k = \sum_{i=0}^{m-1} k_i n^i$$
\end{definition}

Additionally, this protocol uses the coefficients $p_{k,i}$ defined by the polynomials:
$$p_k(x) = \prod_{i=0}^{m-1} f_{i,k_i} = \delta(\ell, k) x^m + \sum_{i=0}^{m-1} p_{k,i} x^i$$

With $f_{i,k_i}$ being the responses in the protocol in \cref{fig:commitment_mat}.
See \cite{one_out_of_many,short_ring_signatures} for the reasoning behind defining $p$ this way.

\begin{theorem}\label{theorem:ring}
	The $\Sigma$--protocol in \cref{fig:ring} is perfectly complete, \ac{SHVZK}, and $(m+1)$-special sound.
\end{theorem}

The proof of \cref{theorem:ring} follows analogous to the proof of Theorem 1 in \cite{triptych}.
% TODO: consider re-iterating full proof here


\begin{figure}[p]
	\centering
	\fbox{\begin{sigmaprotocol}[0.8\textwidth]
		\prover[$\P_{\Rref{rel:ring}}((I, (R_k)_{k=0}^{N-1}), (\ell, r))$]{
			Select random $r_B, \rho_i \from \Z_q \quad \forall i \in \{ 0, .., m-1 \}$ \\
			For all $i \in \{ 0, ..., m-1 \}, j \in \{ 0, ..., n-1 \}$ let $b_{i,j} = \delta(\ell_i, j)$ \\
			Define
			$$B = \Com((b_{i,j})_{i=0,j=0}^{m-1,n-1}, r_B)$$
			Get $(A, C, D) \from \P_{\Rref{rel:commitment_mat}}(B, ((b_{i,j})_{i=0,j=0}^{m-1,n-1}, r_B))$ \\
			For all $i \in \{ 0, ..., m-1 \}$ define
			$$X_i = \sum_{k=0}^{N-1} p_{k,i} R_k + \rho_i I$$
		}{$(A, B, C, D, (X_i)_{i=0}^{m-1})$}

		\verifier[$\V_{\Rref{rel:ring}}((I, (R_k)_{k=0}^{N-1}), (A, B, C, D, (X_i)_{i=0}^{m-1}))$]{
			Get $x \from \V_{\Rref{rel:commitment_mat}}(B, (A, C, D))$
		}{$x$}

		\prover[$\P_{\Rref{rel:ring}}(x)$]{
			Get $(z_A, z_C, (f_{i,j})_{i=0,j=1}^{m-1,n-1}) \from \P_{\Rref{rel:commitment_mat}}(x)$ \\
			Define
			$$z = x^m r - \sum_{i=0}^{m-1} x^i \rho_i$$
		}{$(z, z_A, z_C, (f_{i,j})_{i=0,j=1}^{m-1,n-1})$}

		\verifier[$\V_{\Rref{rel:ring}}(z, z_A, z_C, (f_{i,j})_{i=0,j=1}^{m-1,n-1})$]{
			For all $i \in \{ 0, ..., m-1 \}$ let $f_{i,0} = x - \sum_{j=1}^{n-1} f_{i,j}$
			\par
			Accept if and only if: \\
			$\V_{\Rref{rel:commitment_mat}}(z_A, z_C, (f_{i,j})_{i=0,j=1}^{m-1,n-1})$ accepts and
			\begin{align}
				\sum_{k=0}^{N-1} \left( \left( \prod_{i=0}^{m-1} f_{i,k_i} \right) R_k \right) - \sum_{i=0}^{m-1} x^i X_i = z I
			\end{align}
		}{}
	\end{sigmaprotocol}}
	\caption{$\Sigma$--protocol for \Rref{rel:ring} (ring signature with dynamic public key base $I$)}
\label{fig:ring}\end{figure}


\clearpage


\section{Linkable Ring Signature}
\label{linkable}

Let $\H_\G : \{ 0, 1 \}^* \to \G$ be a secure hash function that hashes to group elements.

\cref{fig:linkable} provides a $\Sigma$--protocol for the linkable ring signature relation using inversion key images, so for the relation \Rref{rel:linkable}:
$$\R{rel:linkable} = \set{(((R_0, ..., R_{N-1}), J), \, (\ell, r))}{R_\ell = rG \; \land \; J = r^{-1} U}$$

\begin{theorem}\label{theorem:linkable}
	The $\Sigma$--protocol in \cref{fig:linkable} is perfectly complete, \ac{SHVZK}, and $(m+1)$-special sound.
\end{theorem}

Proof of \cref{theorem:linkable} follows below in section \ref{proof}.

Applying the Fiat-Shamir heuristic to the protocol in \cref{fig:linkable} gives a linkable ring signature.

This linkable ring signature consists of $m + 7 = \log_n(N) + 7$ group elements and $m(n-1) + 5 = (n-1) \log_n(N) + 5$ field elements.
So for $n = 2$ and assuming group and field elements are of equal size, the signature size scales with regards to $N$ as $2 \log_2(N) + 12$.
Therefore, for any $N > 32$ this novel construction produces shorter signatures than the construction in \cite{triptych} (Triptych).
This is illustrated in \cref{fig:linkable}.
For the concrete size in bytes we assume usage of \texttt{curve25519} \cite{curve25519}, so both group and scalar elements can be represented using 256 bits.

\begin{figure}[h]
	\centering
	\includegraphics[width=0.6\textwidth]{build/linkable.pyplot}
	\caption{Scalability of Triptych+ compared to Triptych}
	\label{fig:linkable}
\end{figure}


\begin{figure}[p]
	\centering
	\fbox{\begin{sigmaprotocol}[0.8\textwidth]
		\prover[$\P'_{\Rref{rel:linkable}}(((R_k)_{k=0}^{N-1}, J), (\ell, r))$]{
			Let $I = \H_\G((R_k)_{k=0}^{N-1})$
			\par
			Select random $s, t, u \from \Z_q$ \\
			Define
			$$\begin{aligned}
				K   &= rG + tI \\
				S_1 &= sG + uI \\
				S_2 &= sJ
			\end{aligned}$$
			For all $k \in \{ 0, ..., N-1 \}$ let
			$$R'_k = R_k - K$$
			Get $(A, B, C, D, (X_i)_{i=0}^{m-1}) \from \P_{\Rref{rel:ring}}((I, (R'_k)_{k=0}^{N-1}), (\ell, -t))$
		}{$(A, B, C, D, K, S_1, S_2, (X_i)_{i=0}^{m-1})$}

		\verifier[$\V'_{\Rref{rel:linkable}}((R_k)_{k=0}^{N-1}, J), (A, B, C, D, K, S_1, S_2, (X_i)_{i=0}^{m-1})$]{
			Let $I = \H_\G((R_k)_{k=0}^{N-1})$
			\par
			For all $k \in \{ 0, ..., N-1 \}$ let
			$$R'_k = R_k - K$$
			Get $x \from \V_{\Rref{rel:ring}}((I, (R'_k)_{k=0}^{N-1}), (A, B, C, D, (X_i)_{i=0}^{m-1}))$
		}{$x$}

		\prover[$\P'_{\Rref{rel:linkable}}(x)$]{
			Get $(z, z_A, z_C, (f_{i,j})_{i=0,j=1}^{m-1,n-1}) \from \P_{\Rref{rel:ring}}(x)$ \\
			Define
			$$\begin{aligned}
				z_G &= rx + s \\
				z_I &= tx + u
			\end{aligned}$$
		}{$(z, z_A, z_C, z_G, z_I (f_{i,j})_{i=0,j=1}^{m-1,n-1})$}

		\verifier[$\V'_{\Rref{rel:linkable}}(z, z_A, z_C, z_G, z_I, (f_{i,j})_{i=0,j=1}^{m-1,n-1})$]{
			Accept if and only if: \\
			$\V_{\Rref{rel:ring}}(z, z_A, z_C, (f_{i,j})_{i=0,j=1}^{m-1,n-1})$ accepts and
			\begin{align}
				xK + S_1 &= z_G G + z_I I \label{eq:linkable:1} \\
				xU + S_2 &= z_G J \label{eq:linkable:2}
			\end{align}
		}{}
	\end{sigmaprotocol}}
	\caption{$\Sigma$--protocol for \Rref{rel:linkable} (linkable ring signature)}
\label{fig:linkable}\end{figure}


\clearpage


\section{Multilayer Linkable Ring Signature}
\label{linkable:multilayer}

\cref{fig:linkable:multilayer:1} and \cref{fig:linkable:multilayer:2} provide a $\Sigma$--protocol for the same multilayer linkable ring signature relation \Rref{rel:linkable:multilayer} as the Triptychs multilayer linkable ring signature protocol.
$$\R{rel:linkable:multilayer} = \set{(((R_{k,u})_{k=0,u=0}^{N-1,d-1}, J), \, (\ell, r_0, ..., r_{d-1}))}{R_{\ell,u} = r_u G \quad \forall u \in \{ 0, ..., d-1 \} \; \land \; J = r_0^{-1} U}$$

\begin{theorem}\label{theorem:linkable:multilayer}
	The $\Sigma$--protocol in \cref{fig:linkable:multilayer:1} and \cref{fig:linkable:multilayer:2} is perfectly complete, \ac{SHVZK}, and $d(m+1)$-special sound.
\end{theorem}

Proof of \cref{theorem:linkable:multilayer} follows below in section \ref{proof}.

Applying the Fiat-Shamir heuristic to the protocol in \cref{fig:linkable:multilayer:1} and \cref{fig:linkable:multilayer:2} gives a multilayer linkable ring signature.

This multilayer linkable ring signature consists of $m + 7 + (d-1) = \log_n(N) + d + 6$ group elements and $m(n-1) + 5 = (n-1) \log_n(N) + 5$ field elements.
So for $n = 2$ and assuming group and field elements are of equal size, the signature size scales with regards to $N$ as $2 \log_2(N) + d + 11$.

As with Triptych, our linkable multilayer ring signature can be used as a drop in replacement for MLSAG/cLSAG in the RingCT construction used by Monero to achieve even more scalable transaction proofs.


\begin{figure}[p]
	\centering
	\fbox{\begin{sigmaprotocol}[0.8\textwidth]
		\prover[$\P'_{\Rref{rel:linkable:multilayer}}(((R_{k,u})_{k=0,u=0}^{N-1,d-1}, J), (\ell, r_0, ..., r_{d-1}))$]{
			For $u \in \{ 1, ..., d-1 \}$ define $V_u = r_u J$ and let
			$$\mu = \H((R_{k,u})_{k=0,u=0}^{N-1,d-1}, J, (V_u)_{u=1}^{d-1})$$
			\par
			For $k \in \{ 0, ..., N-1 \}$ let
			$$R_k = \sum_{u=0}^{d-1} \mu^u R_{k,u}$$
			and let $I = \H_\G((R_k)_{k=0}^{N-1})$
			\par
			Select random $s, t, u \from \Z_q$ \\
			Define
			$$\begin{aligned}
				K   &= \left( \sum_{u=0}^{d-1} \mu^u r_u \right) G + tI \\
				S_1 &= sG + uI \\
				S_2 &= sJ
			\end{aligned}$$
			For all $k \in \{ 0, ..., N-1 \}$ let
			$$R'_k = R_k - K$$
			Get $(A, B, C, D, (X_i)_{i=0}^{m-1}) \from \P_{\Rref{rel:ring}}((I, (R'_k)_{k=0}^{N-1}), (\ell, -t))$
		}{$(A, B, C, D, K, S_1, S_2, (V_u)_{u=1}^{d-1}, (X_i)_{i=0}^{m-1})$}

		\verifier[$\V'_{\Rref{rel:linkable:multilayer}}((R_{k,u})_{k=0,u=0}^{N-1,d-1}, J), (A, B, C, D, K, S_1, S_2, (V_u)_{u=1}^{d-1}, (X_i)_{i=0}^{m-1})$]{
			Let $\mu = \H((R_{k,u})_{k=0,u=0}^{N-1,d-1}, J, (V_u)_{u=1}^{d-1})$
			\par
			For $k \in \{ 0, ..., N-1 \}$ let
			$$R_k = \sum_{u=0}^{d-1} \mu^u R_{k,u}$$
			and let $I = \H_\G((R_k)_{k=0}^{N-1})$
			\par
			For all $k \in \{ 0, ..., N-1 \}$ let
			$$R'_k = R_k - K$$
			Get $x \from \V_{\Rref{rel:ring}}((I, (R'_k)_{k=0}^{N-1}), (A, B, C, D, (X_i)_{i=0}^{m-1}))$
		}{$x$}

		\begin{center}
			continued in \cref{fig:linkable:multilayer:2}
		\end{center}
	\end{sigmaprotocol}}
	\caption{$\Sigma$--protocol for \Rref{rel:linkable:multilayer} (multilayer linkable ring signature)}
\label{fig:linkable:multilayer:1}\end{figure}

\begin{figure}[p]
	\centering
	\fbox{\begin{sigmaprotocol}[0.8\textwidth]
		\begin{center}
			continuation of \cref{fig:linkable:multilayer:1}
		\end{center}

		\prover[$\P'_{\Rref{rel:linkable:multilayer}}(x)$]{
			Get $(z, z_A, z_C, (f_{i,j})_{i=0,j=1}^{m-1,n-1}) \from \P_{\Rref{rel:ring}}(x)$ \\
			Define
			$$\begin{aligned}
				z_G &= \left( \sum_{u=0}^{d-1} \mu^u r_u \right) x + s \\
				z_I &= tx + u
			\end{aligned}$$
		}{$(z, z_A, z_C, z_G, z_I, (f_{i,j})_{i=0,j=1}^{m-1,n-1})$}

		\verifier[$\V'_{\Rref{rel:linkable:multilayer}}(z, z_A, z_C, z_G, z_I, (f_{i,j})_{i=0,j=1}^{m-1,n-1})$]{
			Accept if and only if: \\
			$\V_{\Rref{rel:ring}}(z, z_A, z_C, (f_{i,j})_{i=0,j=1}^{m-1,n-1})$ accepts and
			\begin{align}
				xK + S_1 &= z_G G + z_I I \label{eq:linkable:multilayer:1} \\
				x \left( U + \left( \sum_{u=1}^{d-1} \mu^u V_u \right) \right) + S_2 &= z_G J \label{eq:linkable:multilayer:2}
			\end{align}
		}{}
	\end{sigmaprotocol}}
	\caption{$\Sigma$--protocol for \Rref{rel:linkable:multilayer} (multilayer linkable ring signature)}
\label{fig:linkable:multilayer:2}\end{figure}


\clearpage


\section{Accountable Ring Signature}
\label{accountable}

\cref{fig:accountable} provides a $\Sigma$--protocol for the same accountable ring signature relation \Rref{rel:accountable} as \cite{short_ring_signatures}.
$$\R{rel:accountable} = \set{(((R_k)_{k=0}^{N-1}, J), (\ell, r, u))}{R_\ell = rG \land J = \Enc_Q(R_\ell, u)}$$

\begin{theorem}\label{theorem:accountable}
	The $\Sigma$--protocol in \cref{fig:accountable} is perfectly complete, \ac{SHVZK}, and $(m+1)$-special sound.
\end{theorem}

Proof of \cref{theorem:accountable} follows below in section \ref{proof}.

Applying the Fiat-Shamir heuristic to the protocol in \cref{fig:accountable} gives an accountable ring signature.

This accountable ring signature consists of $m + 8 = \log_n(N) + 8$ group elements (two of those as components of an ElGamal ciphertext) and $m(n-1) + 6 = (n-1) \log_n(N) + 6$ field elements.
So for $n = 2$ and assuming group and field elements are of equal size, the signature size scales with regards to $N$ as $2 \log_2(N) + 14$.
Therefore, it always produces shorter accountable ring signatures than the construction in \cite{short_ring_signatures}.
This is illustrated in \cref{fig:accountable}.

\begin{figure}[h]
	\centering
	\includegraphics[width=0.6\textwidth]{build/accountable.pyplot}
	\caption{Scalability of Triptych+ accountable ring signatures compared to \cite{short_ring_signatures}}
	\label{fig:accountable}
\end{figure}

\begin{figure}[p]
	\centering
	\fbox{\begin{sigmaprotocol}[0.8\textwidth]
		\prover[$\P'_{\Rref{rel:accountable}}(((R_k)_{k=0}^{N-1}, J), (\ell, r, u))$]{
			Let $I = \H_\G((R_k)_{k=0}^{N-1})$
			\par
			Select random $s, t, u_1, u_2 \from \Z_q$ \\
			Define
			$$\begin{aligned}
				K    &= rG + t I \\
				S_1  &= sG + u_1 I \\
				S_2  &= \Enc_Q(sG, u_2)
			\end{aligned}$$
			For all $k \in \{ 0, ..., N-1 \}$ let
			$$R'_k = R_k - K$$
			Get $(A, B, C, D, (X_i)_{i=0}^{m-1}) \from \P_{\Rref{rel:ring}}((I, (R'_k)_{k=0}^{N-1}), (\ell, -t))$
		}{$(A, B, C, D, K, S_1, S_2, (X_i)_{i=0}^{m-1})$}

		\verifier[$\V'_{\Rref{rel:accountable}}((R_k)_{k=0}^{N-1}, J), (A, B, C, D, K, S_1, S_2, (X_i)_{i=0}^{m-1})$]{
			Let $I = \H_\G((R_k)_{k=0}^{N-1})$
			\par
			For all $k \in \{ 0, ..., N-1 \}$ let
			$$R'_k = R_k - K$$
			Get $x \from \V_{\Rref{rel:ring}}((I, (R'_k)_{k=0}^{N-1}), (A, B, C, D, (X_i)_{i=0}^{m-1}))$
		}{$x$}

		\prover[$\P'_{\Rref{rel:accountable}}(x)$]{
			Get $(z, z_A, z_C, (f_{i,j})_{i=0,j=1}^{m-1,n-1}) \from \P_{\Rref{rel:ring}}(x)$ \\
			Define
			$$\begin{aligned}
				z_G &= rx + s \\
				z_I &= tx + u_1 \\
				z_Q &= ux + u_2
			\end{aligned}$$
		}{$(z, z_A, z_C, z_G, z_I, z_Q, (f_{i,j})_{i=0,j=1}^{m-1,n-1})$}

		\verifier[$\V'_{\Rref{rel:accountable}}(z, z_A, z_C, z_G, z_I, (f_{i,j})_{i=0,j=1}^{m-1,n-1})$]{
			Accept if and only if: \\
			$\V_{\Rref{rel:ring}}(z, z_A, z_C, (f_{i,j})_{i=0,j=1}^{m-1,n-1})$ accepts and
			\begin{align}
				xK + S_1 &= z_G G + z_I I \label{eq:accountable:1} \\
				xJ + S_2 &= \Enc_Q(z_G G, z_Q) \label{eq:accountable:2}
			\end{align}
		}{}
	\end{sigmaprotocol}}
	\caption{$\Sigma$--protocol for \Rref{rel:accountable} (accountable ring signature)}
\label{fig:accountable}\end{figure}


\clearpage


\section{Security Proof}
\label{proof}

\begin{definition}\label{def:game}
	Let $\G$ be a group of prime order $q$ and $G$ a generator of $\G$.
	Consider a game between a challenger and a probabilistic polynomial-time player $\A$:
	\begin{itemize}
		\item $\A$ chooses $C \in \G$ and sends it to the challenger.
		\item The challenger chooses $H \in \G$ and sends it to $\A$.
		\item $\A$ chooses and returns $(a, b)$ to the challenger.
	\end{itemize}
	$\A$ wins iff $C = aG + bH$ and $b \ne 0$.
\end{definition}

\begin{lemma}\label{lemma:game}
	If the discrete logarithm problem in $\G$ is hard the game in \cref{def:game} is similarly difficult.
\end{lemma}

\begin{proof}
	To show this we construct an adversary $\A^*$ for the discrete logarithm problem using rewindable black-box access to an efficient adversary $\A$ for this game.
	We consider $\A$ to be an efficient adversary if for any challenge $H \in \G$ it wins with non-negligible probability $\epsilon$.

	To recall, the goal of the discrete logarithm problem adversary $\A^*$ is, given a group element $Y \in \G$, to output $y \in \Z_q$ such that $Y = yG$.

	Adversary $\A^*$ proceeds as follows:
	\begin{itemize}
		\item $\A^*$ gets challenge $Y$ from the challenger and $C$ from $\A$.
		\item $\A^*$ chooses random $z \from \Z^*_q$, queries $\A$ with challenge $zY$ and gets $(a_1, b_1)$ as response.
		With probability of $\epsilon$, $\A$ is successful in responding with $(a_1, b_1)$, such that $b_1 \ne 0$ and
		\begin{equation}\label{eq:game:1}
			C = a_1 G + b_1 zY
		\end{equation}
		If $\A$ did not respond successfully $\A^*$ aborts.
		\item $\A^*$ rewinds $\A$ to just before the previous step.
		\item $\A^*$ chooses random $x \from \Z_q$, queries $\A$ with challenge $xG$ and gets $(a_2, b_2)$ as response.
		With probability of $\epsilon$, $\A$ is successful in responding with $(a_2, b_2)$, such that $b_2 \ne 0$ and
		\begin{equation}\label{eq:game:2}
			C = a_2 G + b_2 xG
		\end{equation}
		If $\A$ did not respond successfully $\A^*$ aborts.

		\NB Since $\A$ was rewound to after having chosen $C$ rather than re-run from the start the $C$ in \cref{eq:game:2} is still the same as in \cref{eq:game:1}.
		\item $\A^*$ computes $y = z^{-1} b_1^{-1}(a_2 - a_1 + b_2 x)$ and sends it to the challenger.
	\end{itemize}

	If $\A$ is able to successfully respond to both queries then $b_1 \ne 0 \implies b_1^{-1}$ exists and from \cref{eq:game:1} and \cref{eq:game:2} we have that
	$$\begin{aligned}
		         & C = a_1 G + b_1 zY = a_2 G + b_2 xG \\
		\implies & Y = z^{-1} b_1^{-1}(a_2 - a_1 + b_2 x) G = yG
	\end{aligned}$$
	So $y$ is the discrete logarithm of $Y$ to $G$ and therefore $\A^*$ also wins the discrete logarithm game.

	If the probability that $\A$ wins the game for any query $H$ is $\epsilon$ the probability that $\A^*$ wins is $\epsilon^2$ since $\A$ needs to win for two distinct queries.
\end{proof}

\begin{corollary}\label{corollary:game}
	From \cref{lemma:game} it follows directly that if a prover is able to somehow prove knowledge of an opening $(a, r)$ to a commitment $C$ such that $C = aG + rH$, where $H$ is a challenge chosen after the commitment $C$ was made, it implies that $r$ must be $0$.
	Because if the prover had an opening with $r \ne 0$ this would be equivalent to winning the game in \cref{def:game}.
\end{corollary}

\begin{proof}[Proof of \cref{theorem:linkable}]
	% completeness
	First we show perfect completeness.
	By the perfect completeness of the $\Sigma$--protocol in \cref{fig:ring} we have that $\V_{\Rref{rel:ring}}$ always accepts.
	Correctness of \cref{eq:linkable:1} and \cref{eq:linkable:2} follows directly by expanding all terms:
	$$\begin{matrix}
	xK + S_1 = x(rG + tI) + sG + uI = (xr + s)G + (xt + u)I = z_G G + z_I I \\
	xU + S_2 = x(rJ) + sJ = (xr + s)J = z_G J
	\end{matrix}$$

	% ZK
	Next, we show that the protocol is \ac{SHVZK}.
	For this, we describe a \ac{SHVZK} simulator.
	The simulator randomly chooses $z_G, z_I \from \Z_q$ and $K \from \G$ and runs the \ac{SHVZK} simulator for $\P_{\Rref{rel:ring}}((I, (R'_k)_{k=0}^{N-1})$, with $R'_k = R_k - K$.
	It computes $S_1 = z_G G + z_I I - xK$ and $S_2 = z_G J - xU$
	By the blinding property of Pedersen commitments $K$ and $S_1$ are indistinguishable from the commitments in a real proof, $J$ is indistinguishable from a real key image by the inverse decisional Diffie-Hellman assumption, and $z_G, z_I$ are uniformly random for both the real proof and the simulated and uniquely determine $S_1$ and $S_2$.

	% soundness
	To show $(m+1)$-special soundness consider responses $z_G, z_I$ and $z'_G, z'_I$ to two distinct challenges $x, x'$.
	From the verification equation \cref{eq:linkable:1} we get
	$$\begin{aligned}
	x K + S_1 &= (x r + s)G + (x t + u)I &= z _G G + z _I I \\
	x'K + S_1 &= (x'r + s)G + (x't + u)I &= z'_G G + z'_I I
	\end{aligned}$$

	Since $I$ was determined by the output of a secure hash function, which we'll model as a random oracle, it follows that the discrete logarithm relation between $G$ and $I$ is unknown to anyone including the prover.
	Therefore we can get two systems of equations, the first:
	$$\begin{aligned}
	x r + s &= z _G \\
	x'r + s &= z'_G
	\end{aligned}$$
	From which we can uniquely extract $r = \frac{z_G - z'_G}{x - x'}$ and subsequently $s = z_G - xr$, and the second:
	$$\begin{aligned}
	x t + u &= z _I \\
	x't + u &= z'_I
	\end{aligned}$$
	From which we can uniquely extract $t = \frac{z_I - z'_I}{x - x'}$ and subsequently $u = z_I - xt$ such that $K = rG + tI$ and $S_1 = sG + uI$.
	Observe that both $K$ and $S_1$ are essentially Pedersen commitments using $I$ in place of $H$ and that $(r, t)$ and $(s, u)$ are openings to $K$ and $S_1$ respectively.

	$(m+1)$-special soundness of the $\Sigma$--protocol in \cref{fig:ring} provides an extractor for values $\ell$ and $t'$ such that $R'_\ell = t' I$.
	$$\begin{aligned}
	&\implies t' I = R'_\ell = R_\ell - K = R_\ell - (rG + tI) \\
	&\implies R_\ell = rG + (t + t') I
	\end{aligned}$$

	So the prover knows an opening $(r, \tau = t + t')$ for commitment $R_\ell$ such that $R_\ell = rG + \tau I$.
	Because $R_\ell$ was part of the input to $\H_\G((R_k)_{k=0}^{N-1}) = I$ it must have been chosen prior to knowledge of $I$ (assuming $\H_\G$ is a secure hash function) so it follows from \cref{corollary:game} that $\tau = 0$ which implies that $R_\ell = rG$.

	Finally, it only remains to show that the key image $J$ was constructed correctly.
	Using the same two distinct challenges $x, x'$ as above we get the following system of equations from the second verification equation \cref{eq:linkable:2}:
	$$\begin{matrix}\begin{aligned}
	x U + S_2 &= (x r' + s')J &= z _G J \\
	x'U + S_2 &= (x'r' + s')J &= z'_G J \\
	\end{aligned} \\ \\
	\Updownarrow \\ \\
	\begin{aligned}
	x r + s' &= z _G \\
	x'r + s' &= z'_G
	\end{aligned}\end{matrix}$$
	From which we can uniquely extract $r' = \frac{z_G - z'_G}{x - x'} = r$ and subsequently $s' = z_G - xr' = z_G - xr = s$ such that $U = r'J = rJ$ and $S_2 = s'J = sJ$.
	$\implies U = rJ \implies J = r^{-1}U$
\end{proof}

\begin{proof}[Proof of \cref{theorem:linkable:multilayer}]
	This proof follows somewhat analogous to the proof of Theorem 3 in \cite{triptych}.
	Using the random oracle model the extractor is allowed to \emph{``program''} the random oracle, so we can treat the value $\mu$ as a challenge variable provided by the extractor.

	Based on the $(m+1)$-special soundness of the sub-protocol in \cref{fig:ring} we can extract a witness of the form $\sum_{u=0}^{d-1} \mu^u r_u$ for any $\mu$ programmed as the random oracle output.
	Doing this for $d$ distinct values of $\mu$ extracts $d$ evaluations of a degree $d-1$ polynomial thus uniquely determines all the coefficients $r_u$ for all $u \in \{ 0, ..., d-1 \}$.

	\NB This would work analogous for $d-1$ independent $\mu_u$ (as in the original Triptych paper) by creating a system of $d$ linear equations with $d$ unknowns thus also extracting all unknowns

	%\todo{Wouldn't this make it $d(m+1)$-special sound? Doesn't the same also apply to the proof of Triptych?}

	With this extraction we get $(r_0 + \sum_{u=1}^{d-1} \mu^u r_u) J = U + \sum_{u=1}^{d-1} \mu^u K_u$ which implies $r_0 J = U$.

	The rest of the proof follows directly from that of \cref{theorem:linkable}.
\end{proof}

\begin{proof}[Proof of \cref{theorem:accountable}]
	This proof follows mostly analogous to that of \cref{theorem:linkable}.

	% completeness
	First we show perfect completeness.
	By the perfect completeness of the $\Sigma$--protocol in \cref{fig:ring} we have that $\V_{\Rref{rel:ring}}$ always accepts.
	Correctness of \cref{eq:accountable:1} and \cref{eq:accountable:2} follows directly by expanding all terms:
	$$\begin{matrix}
	xK + S_1 = x(rG + tI) + sG + uI = (xr + s)G + (xt + u)I = z_G G + z_I I \\
	xJ + S_2 = x(\Enc_Q(R_\ell, u)) + \Enc_Q(sG, u_2) = \Enc((rx + s)G, ux + u_2) = \Enc_Q(z_G G, z_Q)
	\end{matrix}$$

	% ZK
	Next, we show that the protocol is \ac{SHVZK}.
	For this, we describe a \ac{SHVZK} simulator.
	The simulator randomly chooses $z_G, z_I, z_Q \from \Z_q$ and $K \from \G$ and runs the \ac{SHVZK} simulator for $\P_{\Rref{rel:ring}}((I, (R'_k)_{k=0}^{N-1})$, with $R'_k = R_k - K$.
	It computes $S_1 = z_G G + z_I I - xK$ and $S_2 = \Enc_Q(z_G G, z_Q) - xJ$
	By the blinding property of Pedersen commitments $K$ and $S_1$ are indistinguishable from the commitments in a real proof, by the decisional Diffie-Hellman assumption $S_2$ is indistinguishable from ciphertexts used in a real proof, and $z_G, z_I$ are uniformly random for both the real proof and the simulated and uniquely determine $S_1$ and $S_2$.

	% soundness
	Analogous to the proof of \cref{theorem:linkable} we have an extractor for $\ell, r, s, t, u_1$ such that $R_\ell = rG$, $K = rG + tI$, and $S_1 = sG + u_1 I$.
	It only remains to show that the encryption $J$ to $R_\ell$ was constructed correctly.
	Using the two distinct challenges $x, x'$ we get the following system of equations from the second verification equation \cref{eq:accountable:2}:
	$$\begin{aligned}
	x J + S_2 &= \Enc_Q((x r' + s')G, x u + u_2) &= \Enc_Q(z _G G, z _Q) \\
	x'J + S_2 &= \Enc_Q((x r' + s')G, x u + u_2) &= \Enc_Q(z'_G G, z'_Q) \\
	\end{aligned}$$
	From which we can uniquely extract $r' = \frac{z_G - z'_G}{x - x'} = r$, $s' = z_G - xr' = z_G - xr = s$, $u = \frac{z_Q - z'_Q}{x - x'}$, and $u_2 = z_Q - xu$ such that $J = \Enc_Q(r'G, u) = \Enc_Q(R_\ell, u)$
\end{proof}


\clearpage
\printbibliography

\end{document}
